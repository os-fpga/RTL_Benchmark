\section*{Summary}

In short this document describes the assignment of researching and implementing the best point-to-point protocol matching a set of clear requirements. FPGA's are targeted to implement the protocol. The motivation of this work is primarily the increasing demand in data transfer rates. 

This document will start with extensively describing what current point-to-point protocols are used for and what parts they consist of. This explanation will be followed by a survey of currently available protocols will be presented which concludes the Interlaken protocol actually met all of the predefined requirements. These are among others a line rate of 10 Gbps, inclusion of flow control, a determined range distance coverage and CRC. A complete description of the Interlaken protocol will be presented where after an implementation will follow.

The protocol has been realized targeting as much vendor independence as possible. A Xilinx VC707 evaluation board has been provided to realize a proof of concept. Only the transceiver and FIFO's are using IP-cores. The implementation has successfully been tested on a single board using a loop-back fiber and also on two VC707 boards communicating with each other.

The specifications and implementation are published as Free and Open Source on code hosting platforms like~\href{https://opencores.org/}{OpenCores}. Sharing is done with the aim of promoting the dissemination and broad adoption of the designed implementation.


\section*{Samenvatting}
Kortom zal dit document een opdracht beschrijven waarin onderzoek wordt gedaan naar welke van de bestaande point-to-point protocollen het best voldoet aan vooraf opgestelde eisen. Hieronder vallen onder andere een snelheid van 10 Gbps, de aanwezigheid van flow control, een vooraf gedefinieerde af te leggen afstand en CRC. Ook zal hier een implementatie gericht op FPGA's uit volgen. De aanleiding van dit werk is primair de toenemende vraag naar snellere dataoverdracht.

Dit document zal starten met het uitgebreid beschrijven waar de huidige point-to-point protocollen voor gebruikt worden en uit wat voor onderdelen deze zijn opgebouwd. Een uitgebreid onderzoek naar de huidige protocollen zal hierop volgen waaruit geconcludeerd kon worden dat het Interalaken protocol aan alle eisen voldoet. Een complete beschrijving van het Interlaken protocol en de implementatie hiervan zijn tevens beschreven.

Het protocol is gerealiseerd met in gedachten dit zo leverancier onafhankelijk te houden als mogelijk. Een Xilinx VC707 evaluatiebord is gebruikt om mee te testen en eerste concept als bewijs van realiseerbaarheid op te leveren. Alleen de transceiver en FIFO's maken gebruik van IP-cores. Tevens is de implementatie met succes getest op een enkel VC707 board met teruglussen van de glasvezel en ook is het gelukt twee VC707 borden met elkaar te laten communiceren over glasvezel.

De specificaties en implementatie zijn gepubliceerd als gratis en open source op code-hostingplatforms als OpenCores. Het delen word gedaan met achterliggende gedachte om de verspreiding en algehele acceptatie van de ontworpen implementatie te bevorderen.

\newpage
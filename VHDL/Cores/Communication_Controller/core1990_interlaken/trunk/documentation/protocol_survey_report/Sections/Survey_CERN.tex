\section{Traditional CERN protocols}
\label{sec:survey_cern}
CERN of course already had some protocols developed to transfer data between devices. It is very interesting and important for the purpose of this assignment to also look at the existing protocols CERN already has implemented. This way it would also be possible to inspect what their specific pros and cons are and why they are or were adopted. The protocols analyzed are S-Link, Full mode, GBT and the low power variant of GBT.

\subsection{S-Link}
S-Link is a protocol developed in 1995 at CERN and stands for Simple Link Interface. It was developed to connect any layer of front-end electronics to the next layer of read-out electronics. There are multiple implementations of the S-Link available which are also sold as cards. HOLA (High-speed Optical Link for Atlas) is the most recent variant which offers data rates up to 2.0 Gbps.  
There is a some information on the S-Link64 which could achieve a throughput of 6,4Gbps.\cite{S-Link}
In addition to the data transfer, S-Link also offers error detection, a return channel for flow control and for return line signal and even offers a function for self-testing.

\subsection{GBT}
The GBT (GigaBit Transceiver) protocol developed by CERN provides a radiation-hard optical link which can transmit data at speeds of 4,8 Gbps. Generating the GBT frames will be done with an radiation tolerant ASIC.
A single GBT frame consists of 120 bits from which 116 are data/payload. The remaining 4 bits are used as a header which is applied for aligning and to distinguish data and control words. FEC is also included in the GBT protocol. \cite{GBT}

The so called low power variant improves on the energy consumption by reducing this to 25\% in comparison with the original GBT protocol. Data transfer speed stay the same and off course the protocol itself structurally stays the same. However several features are removed to reduce the energy consumption. \cite{GBT_LP}

\subsection{Full mode}
Full mode is a lightweight protocol with a throughput of 9,6 Gbps. Since it uses 8b/10b encoding the maximum user payload is limited at 7,68 Gbps. It is currently applied in the FELIX (Front-End LInk eXchange) system which is part of the Atlas experiment. \cite{FELIX}

\newpage

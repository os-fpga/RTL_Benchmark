\chapter{Architecture}

\section{Block diagram}
The architecture for the full IP core is shown in the Figure~\ref{blockdiagram}. It consists of 2 major parts, the actual
exponentiation core (\verb|mod_sim_exp_core| entity) with a bus interface wrapped around it. In the following sections these 
different blocks are described in detail.\\
\begin{figure}[H]
\centering
\includegraphics[trim=1.2cm 1.2cm 1.2cm 1.2cm, width=10cm]{pictures/block_diagram.pdf}
\caption{Block diagram of the Modular Simultaneous Exponentiation IP core}
\label{blockdiagram}
\end{figure}
\newpage

\section{Exponentiation core}
The exponentiation core (\verb|mod_sim_exp_core| entity) is the top level of the modular simultaneous exponentiation
core. It is made up by 4 main blocks (Figure~\ref{msec_structure}):\\

\begin{itemize}
	\item a pipelined Montgomery multiplier as the main processing unit
	\item RAM to store the operands and the modulus
	\item a FIFO to store the exponents
	\item a control unit which controls the multiplier for the exponentiation and multiplication operations
\end{itemize}

\begin{figure}[H]
\centering
\includegraphics[trim=1.2cm 1.2cm 1.2cm 1.2cm, width=10cm]{pictures/mod_sim_exp_core.pdf}
\cprotect\caption{\verb|mod_sim_exp_core| structure}
\label{msec_structure}
\end{figure}

\subsection{Multiplier}
The kernel of this design is a pipelined Montgomery multiplier. A Montgomery multiplication\cite{MontModMul} allows efficient implementation of a
modular multiplication without explicitly carrying out the classical modular reduction step. Right-shift operations ensure that the length of the (intermediate) results does not exceed $n+1$ bits. The result of a Montgomery multiplication is given by~(\ref{eq:mont}):
\begin{align}\label{eq:mont}
r = x \cdot y \cdot R^{-1} \bmod m \hspace{1.5cm}\text{with } R = 2^{n}
\end{align}
For the structure of the multiplier, the work of \textit{Nedjah and Mourelle}\cite{NedMour} is used as a basis. They show that for large operands ($>$512 bits) the $time\times area$ product is minimal when a systolic implementation is used. This construction is composed of cells that each compute a bit of the (intermediate) result.

Because a fully unrolled two-dimensional systolic implementation would require too many resources, a systolic array (one-dimensional) implementation is chosen. This implies that the intermediate results are fed back to the same same array of cells through a register. A shift register will shift-in a bit of the $x$ operand for every step in the calculation (figure~\ref{mult_structure}). When multiplication is completed, a final check is made to ensure the result is smaller than the modulus. If not, a final reduction with $m$ is necessary.

\textbf{Note:} For this implementation the modulus $m$ has to be uneven to obtain a correct result. However, we can assume that for cryptographic applications, this is the case.


\begin{figure}[H] 
\centering 
\includegraphics[trim=1.2cm 1.2cm 1.2cm 1.2cm, width=15cm]{pictures/mult_structure.pdf}
\caption{Multiplier structure. For clarification the $my$ adder and reduction logic are depicted separately, whereas in practice they are internal parts of the stages. (See Figure~\ref{stage_structure})}
\label{mult_structure}
\end{figure}

\subsubsection{Stage and pipeline structure}
The Montgomery algorithm uses a series of additions and right shifts to obtain the desired result. The main disadvantage is the carry propagation in the adder, and therefore a pipelined version is used. The length of the operands ($n$) and the number of pipeline stages can be chosen before synthesis. The user has the option to split the pipeline into 2 smaller parts so there are 3 operand lengths available during runtime\footnote{e.g. a total pipeline length of 1536 bit split into a part of 512 bit and a part of 1024 bit}.

The stages and first and last cell logic design are presented in Figure~\ref{stage_structure}. Each stage takes in a part of the modulus $m$ and $y$ operand and for each step of the multiplication, a bit of the $x$ operand is fed to the pipeline (together with the generated $q$ signal), starting with the Least Significant Bit. The systolic array cells need the modulus $m$, the operand $y$ and the sum $m+y$ as an input. The result from the cells is latched into a register, and then passed back to the systolic cells for the next bit of $x$. During this pass the right shift operation is
implemented. Each stage thus needs the least significant bit from the next stage to calculate the next step. Final reduction logic is also present in the stages for when the multiplication is complete.

An example of the standard pipeline structure is presented in Figure~\ref{pipeline_structure}. It is constructed using stages with a predefined width. The first cell logic processes the first bit of the $m$ and $y$ operand and generates the $q$ signal. The last cell logic finishes the reduction and selects the correct result. For operation of this pipeline, it is clear that each stage can only compute a step every 2 clock cycles. This is because the stages rely on the result of the next stage.

In Figure~\ref{pipeline_structure_split} an example pipeline design is drawn for a split pipeline. All multiplexers on
this figure are controlled by the pipeline select signal (\verb|p_sel|). During runtime the user can choose which part
of the pipeline is used, the lower or higher part or the full pipeline.

\newpage 
\begin{figure}[H]
\centering
\includegraphics[trim=1.2cm 1.2cm 1.2cm 1.2cm, width=25cm, angle=90]{pictures/sys_stage.pdf}
\caption{Pipeline stage and first and last cell logic}
\label{stage_structure}
\end{figure}
\newpage

\newpage 
\begin{figure}[H]
\centering
\includegraphics[trim=1.2cm 1.2cm 1.2cm 1.2cm, width=25cm, angle=90]{pictures/sys_pipeline_notsplit.pdf}
\caption{Example of the pipeline structure (3 stages)}
\label{pipeline_structure}
\end{figure}
\newpage

\newpage 
\begin{figure}[H]
\centering
\includegraphics[trim=1.2cm 1.2cm 1.2cm 1.2cm, width=22cm, angle=90]{pictures/sys_pipeline.pdf}
\caption{Example of a split pipeline (1+2 stages)}
\label{pipeline_structure_split}
\end{figure}
\newpage


\subsection{Operand RAM and exponent FIFO}
In the core's RAM there is space for 4 operands and 1 modulus. Currently this is instantiated using Xilinx BRAM
primitives, so there is a fixed RAM of 4x1536 bit for the operands + 1536 bit for the modulus available. If using a split
pipeline, it is important that operands for the higher part of the pipeline are loaded into the RAM with preceding
zero's for the lower bits of the pipeline. To store the exponents there is a FIFO of 32 bit wide and 512 deep (also a
Xilinx primitive, FIFO18E1), so it is able to store 2 exponents of each 8192 bit long. Every 32 bit entry has to be
pushed in as 16 bit of $e_0$ for the lower part [15:0] and 16 bit of $e_1$ for the higher part [31:16]. Starting
with the least significant word and ending with the most significant word of the exponents.

\subsection{Control unit}
The control unit loads in the operands and has full control over the multiplier. For single multiplications, it latches in 
the $x$ operand, then places the $y$ operand on the bus and starts the multiplier. In case of an exponentiation, the FIFO is 
emptied while the necessary single multiplications are performed. When the computation is done, the ready signal is 
asserted to notify the system.

\newpage
\subsection{IO ports and memory map}
The \verb|mod_sim_exp_core| IO ports\\
\newline
% Table generated by Excel2LaTeX
\begin{tabular}{|l|c|c|p{8cm}|}
\hline
\rowcolor{Gray}
\textbf{Port} & \textbf{Width} & \textbf{Direction} & \textbf{Description} \bigstrut\\
\hline
\verb|clk|   & 1     & in    & core clock input \bigstrut\\
\hline
\verb|reset| & 1     & in    & reset signal (active high) resets the pipeline, fifo and control logic \bigstrut\\
\hline
\multicolumn{4}{|l|}{\textbf{\textit{operand memory interface}}} \bigstrut\\
\hline
\verb|rw_address| & 9     & in    & operand memory read/write address (structure descibed below) \bigstrut\\
\hline
\verb|data_out| & 32    & out   & operand data out (0 is lsb) \bigstrut\\
\hline
\verb|data_in| & 32    & in    & operand data in (0 is lsb) \bigstrut\\
\hline
\verb|write_enable| & 1     & in    & write enable signal, latches \verb|data_in| to operand RAM \bigstrut\\
\hline
\verb|collision| & 1     & out   & collision output, asserts on a write error \bigstrut\\
\hline
\multicolumn{4}{|l|}{\textbf{\textit{exponent FIFO interface}}} \bigstrut\\
\hline
\verb|fifo_din| & 32    & in    & FIFO data in, bits [31:16] for $e_1$ operand and bits [15:0] for $e_0$ operand \bigstrut\\
\hline
\verb|fifo_push| & 1     & in    & push \verb|fifo_din| into the FIFO \bigstrut\\
\hline
\verb|fifo_nopush| & 1     & out   & flag to indicate if there was an error pushing the word to the FIFO \bigstrut\\
\hline
\verb|fifo_full| & 1     & out   & flag to indicate the FIFO is full \bigstrut\\
\hline
\multicolumn{4}{|l|}{\textbf{\textit{control signals}}} \bigstrut\\
\hline
\verb|x_sel_single| & 2     & in    & selection for x operand source during single multiplication \bigstrut\\
\hline
\verb|y_sel_single| & 2     & in    & selection for y operand source during single multiplication \bigstrut\\
\hline
\verb|dest_op_single| & 2     & in    & selection for the result destination operand for single multiplication \bigstrut\\
\hline
\verb|p_sel| & 2     & in    & specifies which pipeline part to use for exponentiation / multiplication. \bigstrut[t]\\
      &       &       & ``01'' : use lower pipeline part \\
      &       &       & ``10'' : use higher pipeline part \\
      &       &       & ``11'' : use full pipeline \bigstrut[b]\\
\hline
\verb|exp_m| & 1     & in    & core operation mode. ``0'' for single multiplications and ``1'' for exponentiations \bigstrut\\
\hline
\verb|start| & 1     & in    & start the calculation for current mode \bigstrut\\
\hline
\verb|ready| & 1     & out   & indicates the multiplication/exponentiation is done \bigstrut\\
\hline
\verb|calc_time| & 1     & out   & is high during a multiplication, indicator for used calculation time \bigstrut\\
\hline
\end{tabular}%
\newpage
The following figure describes the structure of the Operand RAM memory, for every operand there is a space of 2048 bits 
reserved.\\
\begin{figure}[H]
\centering
\includegraphics[trim=1.2cm 1.2cm 1.2cm 1.2cm, width=15cm]{pictures/msec_memory.pdf}
\caption{RAM structure of the exponentiation core}
\label{RAM_structure}
\end{figure}

\section{Bus interface}
The bus interface implements the register necessary for the control unit inputs to the \verb|mod_sim_exp_core| entity.
It also maps the memory to the required bus and connects the interrupt signals. The embedded processor then has full control
over the core.
\documentclass{gqtekspec}
%%%%%%%%%%%%%%%%%%%%%%%%%%%%%%%%%%%%%%%%%%%%%%%%%%%%%%%%%%%%%%%%%%%%%%%%%%%%%%%%
%%
%% Filename: 	spec.tex
%%
%% Project:	Pipelined Wishbone to AXI coverter
%%
%% Purpose:	This document is a LaTeX description describing how to build
%%		a specification document for the cores within this Pipelined
%%	WB2AXI repository.
%%
%% Creator:	Dan Gisselquist, Ph.D.
%%		Gisselquist Technology, LLC
%%
%%%%%%%%%%%%%%%%%%%%%%%%%%%%%%%%%%%%%%%%%%%%%%%%%%%%%%%%%%%%%%%%%%%%%%%%%%%%%%%%
%%
%% Copyright (C) 2018, Gisselquist Technology, LLC
%%
%% This program is free software (firmware): you can redistribute it and/or
%% modify it under the terms of  the GNU General Public License as published
%% by the Free Software Foundation, either version 3 of the License, or (at
%% your option) any later version.
%%
%% This program is distributed in the hope that it will be useful, but WITHOUT
%% ANY WARRANTY; without even the implied warranty of MERCHANTIBILITY or
%% FITNESS FOR A PARTICULAR PURPOSE.  See the GNU General Public License
%% for more details.
%%
%% You should have received a copy of the GNU General Public License along
%% with this program.  (It's in the $(ROOT)/doc directory, run make with no
%% target there if the PDF file isn't present.)  If not, see
%% <http://www.gnu.org/licenses/> for a copy.
%%
%% License:	GPL, v3, as defined and found on www.gnu.org,
%%		http://www.gnu.org/licenses/gpl.html
%%
%%
%%%%%%%%%%%%%%%%%%%%%%%%%%%%%%%%%%%%%%%%%%%%%%%%%%%%%%%%%%%%%%%%%%%%%%%%%%%%%%%%
%%
%%
\usepackage{import}
\usepackage{bytefield}
\project{Wishbone to AXI}
\title{Specification}
\author{Dan Gisselquist, Ph.D.}
\email{zipcpu (at) gmail.com}
\revision{Rev.~0.1}
\begin{document}
\pagestyle{gqtekspecplain}
\titlepage
\begin{license}
Copyright (C) \theyear\today, Gisselquist Technology, LLC

This project is free software (firmware): you can redistribute it and/or
modify it under the terms of the GNU General Public License as published
by the Free Software Foundation, either version 3 of the License, or (at
your option) any later version.

Some files within this repository have been released under the GNU Lesser
General Public License.  These components may be separated from this repository,
and redistributed or modified under the terms of the Lesser GNU
Public License, again as published by the Free Software Foundation,
either vversion 3 of the License or (at your option) any later
version.  These files will identified as such in their headers.

This program is distributed in the hope that it will be useful, but WITHOUT
ANY WARRANTY; without even the implied warranty of MERCHANTIBILITY or
FITNESS FOR A PARTICULAR PURPOSE.  See the GNU General Public License
for more details, or the GNU Lesser General Public License as appropriate.

You should have received a copy of both the GNU General Public License as
well as the GNU Lesser General Public License along with this repository.
(They are both in the \$(ROOT)/doc directory.  Run make with no target there
if the PDF files are not present.)  If not, see If not, see
\texttt{http://www.gnu.org/licenses/} for a copy.
\end{license}
\begin{revisionhistory}
0.0 & 9/6/2016 & D. Gisselquist & First draft\\\hline
\end{revisionhistory}
% Revision History
% Table of Contents, named Contents
\tableofcontents
\listoffigures
\listoftables
\begin{preface}
The wishbone to AXI controller is born of necessity.  As long as Xilinx's
proprietary IP makes it difficult to access memory, providing only access
via the proprietary AXI bus, some conversion will be necessary for anyone
who wishes to use a wishbone interface.

A special shout out and thanks go to Stephan Wallentowitz, for his first
draft of such a converter, and to Olofk for encouraging me to write this
initial core.

The project has since grown into a general purpose set of both bus bridges
and formal bus properties, to include support for Wishbone, AXI-lite, and
Avalon busses.  The full AXI implementation, together with the bridges
between full AXI and other busses, remains a work in progress.
\end{preface}

\chapter{Introduction}\label{ch:intro}
\pagenumbering{arabic}
\setcounter{page}{1}

%
% Introduction
%
% This section contains the introduction to the core, describing both its
% use and its features.
%


% What is old
% What does the old lack?
% What is new
% What does the new have that the old lacks
% What performance gain can be expected?

\chapter{Architecture}\label{ch:arch}

% This section describes the architecture of the block.  A block level diagram
% should be included describing the top level of the design.

\chapter{Operation}\label{ch:ops}

% This section describes the operation of the core.  Specific sequences, such
% as startup sequences, as well as the modes and states of the block should be
% described.
%

\chapter{Clocks}\label{ch:clocks}

% This section specifies all of the clocks.  All clocks, clock domain passes
% and the clock relations should be described.

% Name | Source |    Rates (MHz)    | Remarks | Description
%               | Max|Min|Resolution|

\chapter{Wishbone Datasheet}\label{ch:wishbone}
\begin{table}[htbp]
\begin{center}
\begin{wishboneds}
Revision level of wishbone & WB B4 spec \\\hline
Type of interface & Slave, Read/Write, pipeline reads supported \\\hline
Port size & Various and configurable \\\hline
Port granularity & 8--bit \\\hline
Maximum Operand Size & Various and configurable\\\hline
Data transfer ordering & (Preserved) \\\hline
Clock constraints & None.\\\hline
Signal Names & \begin{tabular}{ll}
		Signal Name & Wishbone Equivalent \\\hline
		{\tt i\_wb\_clk}	& {\tt CLK\_I} \\
		{\tt i\_wb\_cyc}	& {\tt CYC\_I} \\
		{\tt i\_wb\_stb}	& {\tt STB\_I} \\
		{\tt i\_wb\_we}		& {\tt WE\_I} \\
		{\tt i\_wb\_addr}	& {\tt ADR\_I} \\
		{\tt i\_wb\_sel}	& {\tt SEL\_I} \\
		{\tt i\_wb\_data}	& {\tt DAT\_I} \\
		{\tt o\_wb\_ack}	& {\tt ACK\_O} \\
		{\tt o\_wb\_stall}	& {\tt STALL\_O} \\
		{\tt o\_wb\_data}	& {\tt DAT\_O}
		\end{tabular}\\\hline
\end{wishboneds}
\caption{Wishbone Datasheet}\label{tbl:wishbone}
\end{center}\end{table}

\chapter{I/O Ports}\label{ch:ioports}

% This section specifies all of the core IO ports

% Appendices
% A. May be added to outline different specifications.  (??)


% Index
\end{document}


